\documentclass[12pt,a4paper]{report}

\usepackage[utf8]{inputenc}
\usepackage[T1]{fontenc}


\usepackage{pbox}

% Syntax highlighting for code
\usepackage{listings}

% Language
\usepackage[english]{babel}

% Borders
\usepackage[margin=3.0cm]{geometry}

% Modern style
%\usepackage{lmodern}
%\renewcommand*\familydefault{\sfdefault}

% Use TikZ
\usepackage{tikz}
\usetikzlibrary{positioning,decorations.pathreplacing,shapes}

\usepackage[
	bookmarks=true,
	bookmarksnumbered=true,
	pdfauthor={Stefan Urban},
	pdftitle={Besprechung LK C++},
	colorlinks=false,
	pdfborder={0 0 0},
	linktoc=all
]{hyperref}

% For references
%\usepackage[
%	bookmarks,
%	bookmarksopen=true,
%	bookmarksnumbered=true,
%	pdftitle={Bachelorarbeit: Antennenrotor-Steuerung},
%	colorlinks=false,
%	plainpages=false,% zur korrekten Erstellung der Bookmarks
%	pdfpagelabels,% zur korrekten Erstellung der Bookmarks
%	hypertexnames=false,% zur korrekten Erstellung der Bookmarks
%	linktoc=all
%]{hyperref}

\usepackage{cleveref}

% For correct figure placement
\usepackage{float}

% Includeing pdfs
\usepackage{pdfpages}

% Unnumbered figure captions
\usepackage[center]{caption}

% Correct spacing
\usepackage[onehalfspacing]{setspace}

% Table cell padding
\def\arraystretch{1.5}

% Stick footnotes to the bottom of the page
\usepackage[bottom]{footmisc}

% Correct appendix handling
\usepackage[titletoc]{appendix}
\addto\captionsngerman{\let\appendixtocname\appendixname%
\let\appendixpagename\appendixname}

% Correct units in text
\usepackage{siunitx}
\sisetup{
  locale = DE ,
  per-mode = symbol
}

% Floating figures
\usepackage{wrapfig}

% Header font sizes
\usepackage{sectsty}
\sectionfont{\fontsize{18}{15}\selectfont}
\subsectionfont{\fontsize{13}{15}\selectfont}
\subsubsectionfont{\fontsize{13}{15}\selectfont}


% Distance between paragraphs
\addtolength{\parskip}{2mm}

% Paragraph indents
\setlength{\parindent}{0pt}

% ToC level
\setcounter{secnumdepth}{4}
\setcounter{tocdepth}{4}

% ---------------------------------------------------------------

% Title Page
\title{Besprechung LK C++}
\author{Stefan Urban}


\begin{document}


\textbf{Concept}

\begin{itemize}
	\item Goal determination:\vspace*{-0.3cm}
		\begin{itemize}
			\itemsep0em
			\item Self-localization with SLAM and AMCL, ready-to-use packages that provide map
			\item Aruco-marker detection with RGB camera of kinect-like sensor bar (ready-to-use library)
			\item Goals are positioned in a certain distance before the marker. Those are the actual movement targets.
		\end{itemize}
	\item Robot movement:\vspace*{-0.3cm}
		\begin{itemize}
			\itemsep0em
			\item Receive goals as poses in map.
			\item Hand over goal for the next marker to move base and update it constantly.
			\item If target is currently not available, perform random/search walk.
			\item If target is reached, beep and go on the next one.
		\end{itemize}
\end{itemize}

\textbf{Actual work:}

\begin{itemize}
	\item Familiarize with the robot
	\item Setup launch files, parameters, ...
	\item Record proper map that works flawlessly with AMCL
	\item Use TF to generate proper target pose for each marker
	\item Setup move base
\end{itemize}

\textbf{Characteristic features:}

\begin{itemize}
	\item Dynamic behavior:\\
			We do no have calibration whatsoever! The marker's positions are only rough estimates if the camera detects them from a distance. Especially if the markers are not in the center of the camera's fov. The position get's more accurate the closer we get to the marker.
\end{itemize}

\textbf{Planned / Unfinished work:}

\begin{itemize}
	\item Own driver module, not using move base
	\item Own path planning
\end{itemize}

\end{document}          
